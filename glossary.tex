% glossary
\makeglossaries

\newacronym
  {ash}{ASH}{\gls{ASH}}
\newacronym
  [description={\textit{\acrlong{ccitt}} – Advisory committee to the International Telecommunication Union}]
  {ccitt}{CCITT}{Comité Consultatif International Télégraphique et Téléphonique}
\newacronym
  [description={\textit{\acrlong{cpm}}, an early operating system}]
  {cpm}{CP/M}{Control Program for Microcomputers}
\newacronym
  [description={\textit{\acrlong{cpu}} of a computer system}]
  {cpu}{CPU}{Central Processing Unit}
\newacronym
  [description={\textit{\acrlong{crt}}: technology for building computer screens}]
  {crt}{CRT}{cathode ray tube}
\newacronym
  {dtp}{DTP}{\gls{DTP}}
\newacronym
  [description={A \textit{\acrlong{four-cc}} used as an identifier for a file format or sub-structure}]
  {four-cc}{FourCC}{four character code}
\newacronym
  [description={\gls{IANA}}]
  {iana}{IANA}{Internet Assigned Numbers Authority}
\newacronym
  [description={\textit{\acrlong{iff}}: Precursor to \acrshort{tiff}}]
  {iff}{IFF}{Image File Format}
\newacronym
  [description={\textit{\acrlong{isa}}: The format and semantics of the machine code that a \acrshort{cpu} executes}]
  {isa}{ISA}{instruction set architecture}
\newacronym
  {html}{HTML}{hyper text markup language}
\newacronym
  {m68k}{m68k}{Motorola 68000 line of chips}
\newacronym
  {midi}{MIDI}{Musical Instrument Digital Interface}
\newacronym
  {mime}{MIME}{Multipurpose Internet Mail Extensions}
\newacronym
  {os}{OS}{Operating System}
\newacronym
  {pdf}{PDF}{portable document format}
\newacronym
  {pdl}{PDL}{page description language}
\newacronym
  {png}{PNG}{portable network graphic}
\newacronym
  {ps}{PS}{Adobe PostScript}
\newacronym
  [description={\textit{\acrlong{rom}}: Usually used to build software for an operating system or firmware for a device into a computer.}]
  {rom}{ROM}{read only memory}
\newacronym
  {rtf}{RTF}{rich text format}
\newacronym
  [description={\textit{\acrlong{tiff}}: Container format for images}]
  {tiff}{TIFF}{Tagged Image File Format}
\newacronym
  {tos}{TOS}{The Operating System}
\newacronym
  [description={\textit{\acrlong{wysiwyg}} – A kind of editor that shows the visual output of what is currently being edited}]
  {wysiwyg}{WYSIWYG}{''What You See Is What You Get''}

\newglossaryentry{accessory}
  {name=accessory, plural=accessories, description={Loadable extension to the GEM Disk Operating System}}
\newglossaryentry{ATARI}
  {name={ATARI}, description={\textit{Atari, Inc.} – US-American computer manufacturer and games publisher}}
\newglossaryentry{bit}
  {name={bit}, description={Fundamental unit in computing. Value that is either $0$ or $1$, off or on, false of true, unset or set}}
\newglossaryentry{byte}
  {name={byte}, description={Fundamental unit in computing. A byte is made up of 8 bits and can represent $256$ different values}}
\newglossaryentry{ASH}
  {name={Application Systems /// Heidelberg}, description={Software and Games Publisher based in Heidelberg, Germany. Publisher of Signum! and \textit{Das Signum! Buch}}}
\newglossaryentry{charset}
  {name=charset, description={A collection of glyphs mapped to keys, much like a font but explicitly including other uses like box drawing characters}}
\newglossaryentry{codepoint}
  {name=codepoint, description={Unique number assigned to a character or ligature by the Unicode Standard}}
\newglossaryentry{DRI}
  {name={Digital Research, Inc.}, description={A company founded by Gary Kildall}}
\newglossaryentry{DTP}
  {name={desktop publishing}, description={The process of preparing documents with a computer system}}
\newglossaryentry{encoding}
  {name=encoding, description={A mapping between a sequence of bytes and a sequence of characters}}
\newglossaryentry{hardcopy}
  {name=Hardcopy, description={A term used to describe the process of printing the current content of the computer screen. Roughly equivalent to the modern term \textit{screenshot}. See also: \url{https://www.stcarchiv.de/tos1991/02/hardcopies-mit-mr-print}}}
\newglossaryentry{Header}
  {name={Header}, description={Metadata associated with a document or message that is usually stored at the start or end}}
\newglossaryentry{IANA}
  {name={Internet Assigned Numbers Authority}, description={The standards body in charge of maintaining registries of names and numeric identifiers for internet protocols}}
\newglossaryentry{IFF}
  {name={IFF}, description={Image File Format}}
\newglossaryentry{Media Type}
  {name={Media Type}, description={Also known as MIME type; unique identifier for a file format registered with the IANA}}
\newglossaryentry{Signum!2}
  {name={Signum!2}, description={Second edition of the Signum! word processor and main subject of this paper}}
\newglossaryentry{Unicode}
  {name={Unicode}, description={Standard that is published by the consortium of the same name with the goal of unifying text encoding across all languages}}
\newglossaryentry{word}
  {name={word}, description={Fundamental unit in computing. Describes the default size of memory depending on the processor architecture. For the ATARI, a WORD is 2 bytes or 16 bits}}